\documentclass[12pt]{article}
\usepackage{amsmath,amssymb}
\usepackage{listings}

\setlength{\textheight}{8.5in}
\setlength{\headheight}{.25in}
\setlength{\headsep}{.25in}
\setlength{\topmargin}{0in}
\setlength{\textwidth}{6.5in}
\setlength{\oddsidemargin}{0in}
\setlength{\evensidemargin}{0in}

\newcommand{\myN}{\hbox{N\hspace*{-.9em}I\hspace*{.4em}}}
\newcommand{\myZ}{\hbox{Z}^+}
\newcommand{\myR}{\hbox{R}}

\newcommand{\myfunction}[3]
{${#1} : {#2} \rightarrow {#3}$ }

\newcommand{\myzrfunction}[1]
{\myfunction{#1}{{\myZ}}{{\myR}}}


% Formating Macros
%

\newcommand{\myheader}[4]
{\vspace*{-0.5in}
\noindent
{#1} \hfill {#3}

\noindent
{#2} \hfill {#4}

\noindent
\rule[8pt]{\textwidth}{1pt}

\vspace{1ex} 
}  % end \myheader 

\newcommand{\myalgsheader}[0]
{\myheader{}
{Travelling Salesman Problem \\ Design techniques pseudo codes \\ Emre Can Kucukoglu \\ eckucukoglu@gmail.com \\ https://github.com/eckucukoglu/tsp-solver} {}{}}

% Running head (goes at top of each page, beginning with page 2.
% Must precede by \pagestyle{myheadings}.
\newcommand{\myrunninghead}[2]
{\markright{{\it {#1}, {#2}}}}


\usepackage{algorithm}
\usepackage{algpseudocode}

%%%%%% Begin document with header and title %%%%%%%%%%%%%%%%%%%%%%%%%

\begin{document}

\myalgsheader
\pagenumbering{gobble}
\pagestyle{plain}

\section{Backtracking}

\begin{algorithm}
  \caption{Backtracking design technique for Travelling Salesman Problem}
  \begin{algorithmic}[1]
  
    \Procedure{TSPBacktracking}{$level$, $optcost$, $optx$, $n$}
    \State \textbf{Input:} level, n
    \State \textbf{Output:} optcost, optx 
    \\
    
    \If{lev == n}
        \State C = cost of x
        \If{C \textless optcost}
            \State optcost = C
            \State optx = x
        \EndIf
    \Else
        \State Compute B = B(x)
        \State x[level] = 2
        \While {B \textless optcost and $x[level] \leq n$}
            
            \If{x[level] is distinct from x[1],...,x[level-1]}
                \State TSPBacktracking(level+1, optcost, optx, n)
            \EndIf
            
            \State $x[level] = x[level] + 1$
        \EndWhile
    \EndIf

    \EndProcedure
  \end{algorithmic}
\end{algorithm}

\newpage
\section{Branch \& Bound}
\begin{algorithm}
  \caption{Branch \& Bound design technique for Travelling Salesman Problem}
  \begin{algorithmic}[1]
  
    \Procedure{TSPBranchAndBound}{$level$, $optcost$, $optx$, $n$}
    \State \textbf{Input:} level, n
    \State \textbf{Output:} optcost, optx 
    \State \textbf{Var:} B, C, Count, NextCoord, NextB 
    \\
    
    \If{lev == n}
        \State C = cost of x
        \If{C \textless optcost}
            \State optcost = C
            \State optx = x
        \EndIf
    \Else
        \State Count = 0
        \For{x[level] = 2 to n}
            \If{x[level] is distinct from x[1],...,x[level-1]}
                \State Count = Count + 1
                \State NextCoord[Count] = x[level]
                \State NextB[Count] = B(x)
            \EndIf
        \EndFor
        \State Sort NextCoord according to NextB values
        \State Count = 1
        
        \While {$Count \leq n-level$ and NextB[count] \textless optcost]}
            
            \If{x[level] is distinct from x[1],...,x[level-1]}
                \State x[level] = NextCoord[Count]
                \State TSPBranchAndBound(level+1, optcost, optx, n)
            \EndIf
            \State Count = Count + 1
        \EndWhile
        
        
    \EndIf
    \EndProcedure
  \end{algorithmic}
\end{algorithm}



\end{document}
